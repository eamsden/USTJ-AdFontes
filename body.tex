\section{Urbit Networking: Ad Fontes}

\subsection{In Principio}

\begin{quote}
On Mars, SIN is taken to an extreme. Logically, Urbit is a single
broadcast network - a single big Ethernet wire. Everyone sees everyone
else's packets. You don't send a packet. You ``release'' it. As a matter
of practical optimization, of course, routing is necessary, but it is
opaque to the receiver. If there is a routing envelope, it is stripped
before processing.

An easy way to see this is to compare an Urbit packet, or card, to an IP
packet. An IP packet has a header, which is visible both to the routing
infrastructure and the application stack. The header, of course,
contains the infamous source address. Urbit removes not just the source
address, but also the rest of the header. An Urbit card is all payload -
it has no region seen both by receiver and router.
\end{quote}

\href{https://moronlab.blogspot.com/2010/01/urbit-functional-programming-from.html}{``Urbit:
functional programming from scratch''. \emph{Moron Lab}. January 13,
2010.}

\begin{quote}
In the classical stack, a basic result of protocol design is that you
can't have exactly-once message delivery . To put it in the terms we in
Section 3.2: you can't build a bus on a network. With permanent
networking between solid-state interpreters, this feature is
straightforward. Why? Because two uniformly persistent nodes can
maintain a permanent session
\end{quote}

\begin{quote}
Arvo defines a typed, global, referentially transparent namespace with
the Ames network identity (page 34) at the root of the path. User-level
code has an extended Nock operator that dereferences this namespace and
blocks until results are available. So the Hoon programmer can use any
data in the Urbit universe as a typed constant.
\end{quote}

\href{https://media.urbit.org/whitepaper.pdf}{``Urbit: A Solid-State
Interpreter''. May 26, 2016.}

\subsection{Nunc}

Today there is Ames. Ames provides a \emph{command} protocol with
exactly-once semantics over the wire. Urbit ships can \emph{poke} other
ships and expect eventually once, and only once, action on the poke
provided that at some point in the future both hosts are able to
communicate. Urbit ships can also \emph{watch} a path (wire) on another
ship and receive updates, provided the subscription is not kicked. Both
commands and updates must be acknowledged as a core requirement of the
protocol.

What does not exist is a viable implementation of the typed, global,
referentially transparent namespace. Neither the interface for
application writing, nor the implementation of the stack beneath it,
support programming by binding paths in a namespace and apprising other
ships of such bindings or permitting them to request such bindings.

This document is in aid of the effort to implement such an interface and
networking stack.

\subsection{Observations}

\begin{quote}
\textbf{As a matter of practical optimization, of course, routing is
necessary, but it is opaque to the receiver.}
\end{quote}

This is a massively important observation. Arvo should not ``do
routing.'' It should not be aware of sources or destinations. The
runtime may route. Hoon code kept up to date by Arvo for the runtime's
use may route. Agents may route. Non-Urbit or quasi-Urbit entities on
the network may route. Arvo, a kelvin-versioned artifact which must not
be subject to the uncertainty of external technical development, must
not route.

\subsection{The stack}

From \patp{rovnys-ricfer}:

\begin{itemize}
\item PKI
\item routing and peer discovery
\item transport layer (including packetization)
\item scry resolution layer
\item frontier discovery layer
\item command layer
\end{itemize}

The PKI can be treated for now as a solved problem. Routing and peer
discovery must have presently viable implementations, but nothing which
is being kelvined should constrain future, more Urbit-maximalist
implementations from creation. Likewise the transport layer. Urbit needs
to function both now, over various Ethernets, and millenia from now,
over the Third and Most Glorious Quasiluminal Marterran Subspace
Transponsive. Packetization is not something that should be kelvin
versioned. None of these problems cry out for an immediate solution,
though packetization ought not to churn Arvo's event loop.

\subsubsection{Scry resolution}

The scry resolution layer is where the rubber meets the road. How are
paths bound by remote ships resolved to data? The only possible way is
for some ship (initially the binding ship, and later possibly other
ships and even non-urbit transponders) to communicate the binding. Since
bindings are irrevocable, we call a particular binding an ``oath'' and
we say that a ship will ``avow'' the oath by taking action to inform
others of it.

A ship has no possible way to force its runtime to do anything.
Everything depends on the runtime's operations on a given and current
state of Arvo. (More of the runtime should be written in Hoon, it should
be said.) But a ship can advise or insist that the runtime take some
action. Oaths are immutable but may be avowed many times.

\subsubsection{Frontier discovery}

The frontier is, at this point in time, the limit of what is knowable.
In general we do not know all that is knowable. Many ships have bound
and even avowed oaths which we have not yet heard of, and may never hear
of. But we would like to be able to reliably discover what oaths our
peers will avow to us.

Unfortunately, \textbf{\emph{there is no way to do this which is optimal
for all applications.}} The search for such a mechanism is at the root
of much of the conceptual and design difficulty for Urbit's networking
layer.

We want low-latency, push-based subscriptions. We want low load on large
publishers. There is a fundamental tension here.

The correct approach is to provide an interface for applications to
implement different strategies in this regard. No networked application
can demand absolute upper bounds on delivery latency \emph{and} absolute
reliability.

For example: a chat application might work by eagerly broadcasting
avowals of new chat messages, and then beaconing out reavowals in the
interim between new chat messages on some configurable but acceptable
latency interval. Ten, fifteen, or even thirty seconds of time is not a
steep penalty to pay for a dropped packet in a chat application. Of
course, this penalty is only paid if the entire message is dropped. If
we become aware of a message but we are missing some packets for it, we
can always request (see below) an immediate re-send.

A media streaming application, whether for one-to-many broadcast or
two-way or many-way communication, may never reavow a binding, since a
binding which is known late is no longer relevant.

However, there should of course be a way to request a path from a ship.
Thinking of this in terms of ``which path exactly must I request?'' is
incorrect. The information being communicated is ``I am interested in a
(the latest?) path of this form'', which information ought to be avowed,
as an oath. This permits replication and caching of interest
information, and application-specific but still not onerous handling of
the problem of a dead or silent interest.

\subsubsection{Command layer}

The Ames command layer is, from the view of this system, an application,
if a very general and ubiquitous one. A poke takes the form of an oath
that you wish to poke another ship with some path and data. The
command-layer application insists on re-avowing pokes on a decaying
schedule until they are acknowledged. Acknowledgement is, yet again, an
oath.

\subsection{Routing and discovery, revisited}

Alright then. Urbit is conceptually a broadcast network, but we
necessarily optimize it by routing. So on what basis to we route, and to
where?

Routing must be a decision taken on registration of interest. By what
policy we determine interest may vary, but it should be informed
primarily by avowals of interest by the destination. An exception is
commands which should be routed unless \emph{disinterest} is avowed.

The public Internet is not a broadcast network, and we cannot yet
communicate over a global network where Urbit has its own ethertype and
its own routers. Thus we must be able to select a set of public IPs to
send avowals (our own or others) to.

For this reason, galaxies already live in DNS. Commands and registration
of interest should by default be routed to sponsoring galaxies and be
shared with other galaxies. Galaxies learn IPs of their sponsored stars
by receiving such communication. Perhaps the registeration of interest
is re-avowed on a heartbeat. Similarly, stars learn IPs of their planets
by receiving registrations of interest. Any retransmitted avowal can
have several lanes attached in the packet header (stripped before
offering to Arvo) by which the avowing ship can be reached. Commands can
be routed to the IP of the commanded ship, if known, its sponsor's IP,
if known, or its sponsor's sponsor's IP. More general oaths are checked
against registered interest (which may be expired if not re-avowed) and
sent along known lanes to interested ships.

(In an Urbit maximalist future, BGP-type protocols between Urbit-aware
routers will likely permit link-shaped routing of such avowals as well
as sponsorship-shaped routing, which can reasonably exist now.)

Local networks, by contrast, generally permit and even depend on
broadcast behavior. The Address Resolution Protocol is fundamental to
the operation of local IP subnets over lower-layer networks. A packet is
broadcast which requests a MAC address for an IP, and provides a MAC
address to reply to. At higher layers this is replicated. Peer discovery
for local networks at the OS layer (DHCP, zeroconf networking),
shared-hardware layer (printers and screens) and application layer
(media) rely on on broadcast announcements which receive addressed
replies. Commands and interest registrations could be broadcast on local
networks to discover peers either directly, or via Urbit-aware routers.
